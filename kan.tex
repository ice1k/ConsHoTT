\section{Kan operations}
\label{sec:kan}

In~\cref{subsec:coe}, we have seen an operation on paths,
that is \textsf{coe}, which returns a path~\footnote
{It's not actually a path in Arend, but a function over intervals.
  We can say it's \textit{almost} a path.}.
Since it returns a path, we may wonder how is the returned path
represented internally.
Ideally, we want \textsf{coe} to be a built-in but reducible function,
which means that it \textbf{computes} when fully applied.
Unfortunately, \textsf{coe} in Arend only computes in certain hard-coded cases,
which is not ideal.
There is a more general operation in CTT and CCTT which allows computation
on path operations, namely \textit{Homogenous composition}.

The idea is in one sentence:

\begin{displayquote}
  For any n-dimensional cube, it has $2 \times n$ faces.
  If we can construct $2 \times n - 1$ of them,
  the last one face can be obtained by doing a homogenous composition.
\end{displayquote}

In case it's two-dimensional, the cube becomes a square,
and it has four faces, which are all one-dimensional paths.
If we can have three of these paths,
we can obtain the last path via homogenous composition.
We can graph this process, as in~\cref{fig:simple-comp}
(assuming $\vdash a, b, c, d : A$).

\begin{figure}
\begin{center}
  \begin{tikzpicture}[node distance=1.5cm]
    \node(1) {$a$};
    \node(2) [right=4cm of 1] {$b$};
    \node(4) [below of=1] {$c$};
    \node(3) [below of=2] {$d$};
    \draw (1) -- (2) node[midway,above] {Obtained by composition};
    \draw (1) -- (4) node[midway,left ] {$p : a =_A c$};
    \draw (3) -- (4) node[midway,below] {$r : c =_A d$};
    \draw (2) -- (3) node[midway,right] {$q : b =_A d$};
  \end{tikzpicture}
\end{center}
\caption{Simple composition}
\label{fig:simple-comp}
\end{figure}

This looks very similar to~\cref{fig:filler}
if we substitute $a$ with $b$ and $b, c, d$ with $a$.
We can also prove path composition by substituting $a$ with $c$.

Here's the surface syntax of homogenous composition.
In~\cite{CCHM}, there are two basic structures that the composition operation
is based on:

\begin{align*}
  \varphi &= (i=\textsf 0) \mid (i=\textsf 1)
  & \xtag \\
  u &= [ \; \varphi_1 \mapsto t_1,
      \dots, \varphi_n \mapsto t_n \; ]
  & \xtag
\end{align*}

$\varphi$ is called the \textit{Face lattice},
which stands for a constraint that specifies a face.
$u$ is a list of face-term pair $\varphi \mapsto t$,
where each pair specifies a term $t$ for a face $\varphi$.
Thus we can describe an open shape via $u$.

The composition operation is defined as following:

\newcommand{\comp}{\textsf{comp}}

\[
  \cfrac
  {\Gamma, i : \mathbb I \vdash A \quad
    \Gamma, \varphi, i : \mathbb I \vdash u : A \quad
    \Gamma \vdash a_0 : A(\textsf 0)[ \varphi \mapsto u(\textsf 0) ]}
  {\Gamma \vdash \comp^i~A~[ \varphi \mapsto u ]~a_0 :
    A(\textsf 1)[ \varphi \mapsto u(\textsf 1) ]}
\]

In naturally language, given
$A$ -- a type indexed by $\mathbb I$,
$a_0$ -- a term for the bottom face of the cube,
$u$ -- a list of face-term pairs representing all the faces except
$a_0$ and the top-missing face.
Then, $\comp^i~A~[ \varphi \mapsto u ]~a_0$ produces the top face.

In CCTT, there is another variation of $\varphi$:

\[
  \varphi = \dots \mid (i = j)
  \xtag
\]

This face specifies a diagonal,
giving us more computation rules.
