\subsection{Transport}
\label{subsec:coe}

The idea of \textit{type-safe coercion} is like
casting a variable of type $A$ to type $B$ by providing a proof
of $A =_{\mathcal U} B$:

\[
  \textsf{coe} : A =_{\mathcal U} B \rightarrow A \rightarrow B
  \xtag \label{eqn:coe-a-b-id}
\]

Bringing in the idea of path types,
assuming $p : A =_{\mathcal U} B$, by~\ref{eqn:path-app}
we know that $p~\textsf 0 \equiv A$ and
$p~\textsf 1 \equiv B$,
so~\ref{eqn:coe-a-b-path} is effectively the same
as~\ref{eqn:coe-a-b-id}:

\[
  \textsf{coe} : (p : A =_{\mathcal U} B) \rightarrow p~\textsf 0
  \rightarrow p~\textsf 1
  \xtag \label{eqn:coe-a-b-path}
\]

The return type of \textsf{coe} can be generalized over
the interval on the input path:

\[
  \textsf{coe} : (p : A =_{\mathcal U} B) \rightarrow p~\textsf 0
  \rightarrow (i : \mathbb I) \rightarrow p~i
  \xtag \label{eqn:coe-a-b-path-gen}
\]

\ref{eqn:coe-a-b-path-gen} is known as the ``generalized transport''
operation, denoted as \textsf{coe} in Arend and CCTT,
and \AgdaFunction{transp} in CTT.
We use \textsf{coe} to refer to this general concept for consistency.
And by the way,
Arend's \textsf{coe} takes function over interval instead
of a path as its first argument:

\[
  \textsf{coe} : (p : I \rightarrow \mathcal U)
  \rightarrow p~\textsf 0
  \rightarrow (i : \mathbb I) \rightarrow p~i
  \xtag
\]

We can state that $\textsf{coe}~p~a~\textsf 0$ will give $a$ back
(i.e. reduce to $a$).

Right now there is neither nontrivial path nor path on types,
while transport along identity paths is just a fancy version
of the identity function, thus we do not need to worry about the
computation of \textsf{coe} yet.
We further discuss \textsf{coe} in~\cref{sec:ua}.

In CTT and CCTT, we cannot transport along paths between paths,
but it's not the case in Arend.
Arend uses \textsf{coe} to prove path symmetry and transitivity.
Here's a proof, assuming three inhabitants $a, b, c$ of type $A$
and the trivial path $\refl_a : a =_A a$:

\[
  \cfrac{
    \Gamma \vdash p : a =_A b
  }{
    \Gamma \vdash (\textsf{coe}
    (\lambda i. (p~i =_A a))~\refl_a~\textsf 1)
    : b =_A a
  }
  \xtag \label{eqn:arend-proof-sym}
\]

In Arend's concrete syntax, path application is not
$p~i$ but $p~@~i$ and lambda abstraction is not $\lambda x.y$ but
$\backslash{}\textsf{lam}~x \Rightarrow y$.
Therefore the concrete syntax of~\ref{eqn:arend-proof-sym} in Arend is:

\begin{minted}{arend-lexer.py:ArendLexer -x}
  \func sym {A : \Type} {a b : A} (p : a = b) : b = a =>
    coe (\lam i => p @ i = a) idp right
\end{minted}

What happened in~\ref{eqn:arend-proof-sym} is that we transported
along the following interval lambda:

\[
  f = \lambda i. (p~i =_A a)
  \xtag
\]

As $p : a =_A b$, we have
$(f~\textsf 0) \equiv (p~\textsf 0 =_A a) \equiv (a =_A a)$ and
$(f~\textsf 1) \equiv (p~\textsf 1 =_A a) \equiv (b =_A a)$,
so $f$ is similar to a path $(a =_A a) =_{\mathcal U} (b =_A a)$.
By providing a proof $\refl_a$ of $a =_A a$,
we obtain the proof of the right hand side of $f$, which is $b =_A a$.
The proof of transitivity is similar, as in~\ref{eqn:arend-proof-trans}.
Analyzing how it works is left as an exercise.

\[
  \cfrac{
    \Gamma \vdash p : a =_A b
    \quad
    \Gamma \vdash q : b =_A c
  }{
    \Gamma \vdash (\textsf{coe}
    (\lambda i. (a =_A q~i))~p~\textsf 1)
    : a =_A c
  }
  \xtag \label{eqn:arend-proof-trans}
\]

It looks like the following in Arend:

\begin{minted}{arend-lexer.py:ArendLexer -x}
  \func trans {A : \Type} {a b c : A}
              (p : a = b) (q : b = c) : a = c =>
    coe (\lam i => a = q @ i) p right
\end{minted}