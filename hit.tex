By the above constructive path type,
we can extend inductive types with path constructors.
Recall that a constructor of an inductive type $T$ is
similar to a function whose return type is $T$,
but does not reduce.

Inductive types with path constructors are called
\textit{higher inductive types} (hereafter as HIT),
which is discussed in Chapter 6 of the HoTT Book.
Here's a simple example of HIT,
where the name \textsf{Seg} stands for \textit{Segment}:

\[
  \vdash \textsf{Seg} \ \textbf{type}
  \xtag
\]
\[
  \vdash \textsf{left} : \textsf{Seg}
  \xtag
  \quad
  \vdash \textsf{right} : \textsf{Seg}
  \xtag
\]
\[
  \vdash \textsf{inner} :
  \textsf{left} =_{\textsf{Seg}} \textsf{right}
  \xtag
\]

The type \textsf{Seg} has two \textbf{point constructors}
\textsf{left} and \textsf{right}, with a \textbf{path constructor}
\textsf{inner} whose two endpoints are \textsf{left} and \textsf{right}.

Path constructors constraints the operations defined on HITs.

Higher inductive types are similar in CTT and CCTT,
while Arend is somehow different.

\TODO