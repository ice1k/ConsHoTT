To define Path constructively,
we may get some inspiration from its topological definition.

Open-up the \href{https://ncatlab.org/nlab/show/path}{``path'' segment in nLab},
there is a mathematical definition of a path, written as:

\[
  \mathbb I \rightarrow X
  \xtag
\]

Topologically, a path in a space $X$ is a continuous map
from an interval (denoted $\mathbb I$) to $X$.
Type theoretically, $\mathbb I$ and $X$ are supposed to be types,
while $\rightarrow$ forms function types.
As paths represent a relation \textit{between} two terms
(endpoints, but type theoretically),
these two terms should show up in the path type as well
(similar to MLTT identity type).
Therefore the formation rule for path types is naturally:

\[
  \cfrac{
    \Gamma \vdash X \ \textbf{type}
    \quad
    \Gamma \vdash a : X
    \quad
    \Gamma \vdash b : X
  }{\Gamma \vdash a =_X b \ \textbf{type}}
  \xtag
\]

Up to the time where this note is written,
everyone tries to define constructive HoTT defines path types this way.
They have different introduction and elimination rules,
but all of their introduction rules are
based on the interval type $\mathbb I$.

We can also define heterogeneous path types
(path between two terms of different types)
by changing the type $X$ into a type family $f$,
indexed by the interval type $\mathbb I$:

\[
  \cfrac{
    \Gamma \vdash X \ \textbf{type}
    \quad
    \Gamma \vdash a : X
    \quad
    \Gamma \vdash b : X
    \quad
    \Gamma \vdash f : \mathbb I \rightarrow X
  }{\Gamma \vdash a =_f b \ \textbf{type}}
  \xtag
\]

\subsection{Interval types}

The interval type has very simple formation rule
and introduction rule:

\[
  \vdash \mathbb I\ \textbf{type}
  \xtag \quad
  \vdash \textsf 0 : \mathbb I
  \xtag \quad
  \vdash \textsf 1 : \mathbb I
  \xtag
\]

The programming language Arend~\cite{Arend} uses a different notation
(\textsf{left} instead of \textsf 0, \textsf{right} instead of \textsf 1)
for interval endpoints.
We will still use \textsf 0 and \textsf 1 when talking
about Arend for consistency.

The interval type do not yet have an elimination rule,
so we cannot have a predicate on an interval.

By this definition of interval, the path type can
have the following introduction rule
(definitional equality between term $a$ and $b$
is denoted as $a \equiv b$,
usually implemented via conversion checking or normalization):

\[
  \cfrac{
    \Gamma \vdash a =_X b \ \textbf{type}
    \quad
    \Gamma, i : \mathbb I \vdash t : X
    \quad
    (\lambda i. t) \ \textsf 0 \equiv a
    \quad
    (\lambda i. t) \ \textsf 1 \equiv b
  }{
    \Gamma \vdash \lrangle i t : a =_X b
  }
  \xtag
\]

Heterogeneously:

\[
  \cfrac{
    \Gamma \vdash a =_F b \ \textbf{type}
    \quad
    \Gamma, i : \mathbb I \vdash t : F \ i
    \quad
    (\lambda i. t) \ \textsf 0 \equiv a
    \quad
    (\lambda i. t) \ \textsf 1 \equiv b
  }{
    \Gamma \vdash \lrangle i t : a =_F b
  }
  \xtag
\]

The above definition is used in Cubical Type Theory~\cite{CCHM,CHM}
(hereafter as CTT), Cartesian Cubical Type
Theory~\cite{CCTT,CCTT2,CHTT} (hereafter as CCTT).
There are three usable implementations of CTT described in~\cite{CHM},
including \href{https://github.com/mortberg/cubicaltt}{cubicaltt},
\href{https://github.com/molikto/mlang}{mlang} and
\href{https://agda.readthedocs.io/en/v2.6.0.1/language/cubical.html}
{Cubical Agda}~\cite{CubicalAgda}.
For CCTT, there are two, including
\href{https://github.com/RedPRL/redtt}{redtt}
and \href{https://github.com/mortberg/yacctt}{yacctt},
implementing different models.

The syntax is intentionally made similar to a lambda abstraction,
as the introduction rule is the same as lambda abstraction with
additional two definitional equalities required.
The elimination rule for a path is therefore similar to an application,
with two additional reduction rules -- applying an interval $i$ to
an arbitrary term whose type is known to be a path type $a =_X b$
will reduce to $a$ if $i \equiv \textsf 0$ or $b$ if $i \equiv \textsf 1$:

\[
  \cfrac{
    \Gamma \vdash p : a =_X b
  }{
    \Gamma \vdash p\ \textsf 0 \equiv a
    \quad
    \Gamma \vdash p\ \textsf 1 \equiv b
  }
  \xtag \label{eqn:path-app}
\]

Rule~\ref{eqn:path-app} holds even if $p$ is a neutral term,
or a contructor (in case there's path constructors,
introduced in~\cref{sec:hit}).
Therefore constructor application can be redex as well.

CTT and CCTT (and many other variations) have different primitive
operations defined for the interval type,
we discuss this later in~\cref{sec:kan}.

Arend on the other hand defines a primitive operator \textsf{path}
as the introduction rule for path:

\[
  \cfrac{
    \Gamma \vdash X \ \textbf{type}
    \quad
    \Gamma \vdash t : \mathbb I \rightarrow X
  }{
    \Gamma \vdash \textsf{path} \ t : (t \ \textsf 0) =_X (t \ \textsf 1)
  }
  \xtag
\]

The elimination of paths is still similar to CTT or CCTT.

\subsection{Path properties}
\input{latex/path-properties}

